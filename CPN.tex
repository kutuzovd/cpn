\documentclass[unicode, notheorems]{beamer}
%\usepackage{pscyr}
\usepackage{cmap}
\usepackage[T2A]{fontenc}
\usepackage[utf8]{inputenc}
\usepackage[english,russian]{babel}
\usepackage{amsthm,amsmath,amssymb}
\usepackage{graphicx}
%\usepackage[figuresright]{rotating}
% пакет hyperref используется для гиперссылок (сайт, почтовый адрес и т.п.)
% для Beamer опции не используют
\usepackage{hyperref}
\usepackage{color}

\usepackage{tikz}
\usetikzlibrary{arrows,decorations.pathmorphing,shapes,snakes,automata,backgrounds,positioning,fit,petri}

%для отображения нумерации в рисунках и таблицах
\setbeamertemplate{caption}[numbered]

%\usetheme{Berlin}
\usetheme{Szeged}
% Это только часть возможных стилей, некоторые из них могут содержать
% дополнительные опции
% default, AnnArbor, Antibes, Bergen, Berkeley, Berlin, Boadilla, CambridgeUS,
%Copenhagen, Darmstadt, Dresden, Frankfurt, Goettingen, Hannover,
%Ilmenau, JuanLesPins, Luebeck, Madrid, Malmoe, Marburg, Montpellier,
%PaloAlto, Pittsburgh, Rochester, Singapore, Szeged, Warsaw.

\title{Моделирование дскретно-событийных систем}
\subtitle{Сети Петри}
\author[Д.В.Кутузов]{Кутузов Денис Валерьевич}
\institute[АГТУ]{<<Астраханский государственный технический университет>>}
\date{Астрахань, 2018~г.}


%Стили рисования для сети Петри (места и переходы)
%   \tikzset{
%   place/.style={
%     circle,
%     thick,
%     draw=black,
%     %fill=blue!20,
%     minimum size=6mm
%   },
%   red_place/.stile={
%     place,
%     draw=red!75,
%     fill=red!20
%   },
%   transition/.style={
%     rectangle,
%     thick,
%     fill=black,
%     minimum width=8mm,
%     inner ysep=2pt
%     }
%   }            

\tikzstyle{place}=[circle,thick,draw=black!75,minimum size=6mm]
\tikzstyle{red place}=[place,draw=red!75,fill=red!20]
\tikzstyle{transition}=[rectangle,thick,draw=black,fill=black,minimum height=8mm]

\begin{document}

\begin{frame}
\titlepage
\end{frame}

% \begin{frame}[allowframebreaks]
% \tableofcontents
% \end{frame}

\section{Сети Петри}
%\subsection{Уровни технического творчества}

\begin{frame}{Историческая справка}
Сети Петри изначально разрабатывались для моделирования систем с параллельными взаимодействующими компонентами.
\newline
\begin{columns}
  \begin{column}{.3\linewidth}  % задаём ширину первой колонки
    \begin{figure}[!h]
    \centering \pgfimage[width=0.8\textwidth]{Carl_adam_petri}
    \end{figure}
   \end{column}
  \begin{column}{.6\linewidth}  % задаём ширину второй колонки
   Впервые были предложены Карлом Адамом Петри в 1962 году в докторской диссертации <<Связь автоматов>> 
(Kommunikation mit Automaten). В этой работе он сформулировал основные понятия теории связи асинхронных 
компонент вычислительной системы  
  \end{column}
\end{columns}
\end{frame}
 


\begin{frame}{Возможности моделирования}
Сети Петри использовуюся для моделирования динамических дискретно-событийных систем:
\begin{enumerate}
  \item Вычислительных систем
  \item Автоматов
  \item Процессов
  \item Протоколов
  \item Некоторых СМО (таймированные сети)
  \item Функционирования служб и отделов организации
  \item Организации технологических цепочек
  \item В робототехнике (для моделирования поведения роботов)
\end{enumerate}
\end{frame}



\begin{frame}{Неформальное определение}
Сеть Петри состоит из мест (place), переходов (transition), соединяющих их дуг (edge) и фишек или маркеров (token). 

\begin{figure}[!h]
\begin{tikzpicture}[node distance=1.3cm,>=stealth',bend angle=45,auto]
    \node [place,tokens=1] (p1) [label=above:$p_{1}$, label=below:место] {};
    
    \node [transition] (t1) [right=of p1, label=above:$t_{1}$, label=below:переход] {}
      edge [pre] (p1);
    
    \node [place] (p2) [right=of t1, label=above:$p_{2}$, label=below:место] {}
      edge [pre] (t1);
\end{tikzpicture}
\end{figure}
\end{frame}


\begin{frame}{Функционирование сети Петри}
Срабатываение перехода $t_{i}$ представляет собой удаление одного (или нескольких) токенов из
каждой позиции $p_{i}$, если имеется исходящая дуга из $p_{i}$ в $t_{i}$, и добавление токена в
позицию $p_{j}$, если имеется заходящая дуга из  $t_{i}$  в $p_{j}$.\\

\begin{columns}
  \begin{column}{.45\linewidth}  % задаём ширину первой колонки
    До срабатывания перехода $t_{1}$ (начальная маркировка):
    \begin{figure}[!h]
   \begin{tikzpicture}[node distance=1.3cm,>=stealth',bend angle=45,auto]
    \node [place,tokens=1] (p1) [label=above:$p_{1}$] {};
    
    \node [transition] (t1) [right=of p1, label=above:$t_{1}$] {}
      edge [pre] (p1);
    
    \node [place] (p2) [right=of t1, label=above:$p_{2}$] {}
      edge [pre] (t1);
\end{tikzpicture}
\end{figure}
   \end{column}
  \begin{column}{.5\linewidth}  % задаём ширину второй колонки
   После срабатывания перехода $t_{1}$ (конечная маркировка):
\begin{figure}[!h]
\begin{tikzpicture}[node distance=1.3cm,>=stealth',bend angle=45,auto]
    \node [place] (p1) [label=above:$p_{1}$] {};
    
    \node [transition] (t1) [right=of p1, label=above:$t_{1}$] {}
      edge [pre] (p1);
    
    \node [place,tokens=1] (p2) [right=of t1, label=above:$p_{2}$] {}
      edge [pre] (t1);
\end{tikzpicture}
\end{figure}
  \end{column}
\end{columns}
\end{frame}





\begin{frame}{Функционирование сети Петри}

\begin{columns}
  \begin{column}{.5\linewidth}  % задаём ширину первой колонки
    До срабатывания перехода $t_{1}$ (начальная маркировка):
\begin{figure}[!h]
\begin{tikzpicture}[node distance=1.3cm,>=stealth',bend angle=45,auto]
    \node [place,tokens=1] (p1) [label=above:$p_{1}$] {};
    
    \node [place,tokens=1] (p2) [below=of p1, label=above:$p_{2}$] {};
        
    \node [transition] (t1) [below right=0.25cm and 2cm of p1, label=above:$t_{1}$] {}
      edge [pre,bend right] (p1)
      edge [pre,bend left] (p2);
    
    \node [place] (p3) [right=of t1, label=above:$p_{3}$] {}
      edge [pre] (t1);
\end{tikzpicture}
\end{figure}
   \end{column}
  
  \begin{column}{.5\linewidth}  % задаём ширину второй колонки
   После срабатывания перехода $t_{1}$ (конечная маркировка):
\begin{figure}[!h]
\begin{tikzpicture}[node distance=1.3cm,>=stealth',bend angle=45,auto]
    \node [place] (p1) [label=above:$p_{1}$] {};
    
    \node [place] (p2) [below=of p1, label=above:$p_{2}$] {};
        
    \node [transition] (t1) [below right=0.25cm and 2cm of p1, label=above:$t_{1}$] {}
      edge [pre,bend right] (p1)
      edge [pre,bend left] (p2);
    
    \node [place,tokens=1] (p3) [right=of t1, label=above:$p_{3}$] {}
      edge [pre] (t1);
\end{tikzpicture}
\end{figure}
  \end{column}
\end{columns}
\end{frame}


\end{document}
	